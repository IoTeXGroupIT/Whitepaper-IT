\section{Blockchain}
La tecnologia della Blockchain è stata introdotta nel 2008 e la sua prima implementazione, ovvero Bitcoin, è stata introdotta un anno dopo, nel 2009, pubblicata nel documento \emph{Bitcoin: A Peer-to-Peer Electronic Cash System} \cite{c21} di Satoshi Nakamoto (pseudonimo). Essenzialmente, la blockchain è un database transazionale distribuito, condiviso tra tutti i nodi partecipanti nella rete. Questa è la principale innovazione tecnica di Bitcoin, ed agisce come un registro pubblico per le transazioni. Ogni nodo nel sistema ha una copia completa dello stato attuale della blockchain, che contiene ogni transazione che sia mai stata eseguita. Ogni blocco della blockchain contiene l'hash del blocco precedente, il che collega i due blocchi insieme. Tutti i nodi collegati tra loro diventano una blockchain.

\subsection{Ingredienti}
Una blockchain può essere percepita come un continuum tetra-dimensionale avente tre livelli orizzontali composti da transazioni e blocchi, consenso, e interfaccia di calcolo; e l'unico livello verticale di amministrazione (\emph{governance}).

\subsubsection{Transazioni e blocchi}
Trovandosi al livello orizzontale più in basso, le transazioni firmate vengono trasmesse tra tutti i nodi, mentre i blocchi vengono generati solo dai nodi completi (\emph{full nodes}). Questa è la base della blockchain, dove il trasferimento di beni digitali (e dunque il valore a loro associato) e la sicurezza degli account si ottengono attraverso primitive come la firma a curva ellittica, la funzione hash e il Merkle tree.

\subsubsection{Consenso}
Il livello orizzontale intermedio mostra la natura peer-to-peer della blockchain, dove tutti i nodi all'interno della rete raggiungono il consenso su tutti gli stati interni della catena attraverso tecniche come Proof of Work (PoW), Proof of Stake (PoS) e le loro varianti; Byzantine fault tolerance (BFT) e le sue varianti, \emph{etc}. Il livello di consenso interessa principalmente la scalabilità. Il PoW viene tipicamente considerato meno scalabile rispetto al PoS. Inoltre, questo livello ha un forte impatto sulla sicurezza in termini di doppia spesa ed altri attacchi che si concentrano sul modificare gli stati della blockchain in modi inattesi.

\subsubsection{Interfaccia di calcolo}
I primi due livelli orizzontali realizzano \emph{la forma} di una blockchain, mentre il livello di interfaccia di calcolo è fondamentale per garantire l'\emph{utilità} di una blockchain, il che comprende l'estendibilità ed l'usabilità. Ad esempio, Ethereum ha implementato gli smart contract per consentire la programmabilità, in modo da poter disporre di un \emph{"computer globale"} distribuito per l'esecuzione dei termini di un contratto. Anche il sidechain, insieme con il mining congiunto, sono stati sviluppati in modo intensivo per supportare la programmabilità. All'interno di protocolli di secondo livello come la rete Raiden \cite{c25}, è stato sviluppato il canale di stato per estendere la scalabilità della blockchain. Inoltre, anche gli strumenti, gli SDK, i framework e le interfacce grafiche sono estremamente importanti per l'usabilità. Il livello di interfaccia di calcolo offre agli sviluppatori la possibilità di sviluppare app decentralizzate (DApps), una funzionalità essenziale per rendere la blockchain utile e di valore.

\subsubsection{Amministrazione}
Come accade per gli organismi viventi, le blockchain di maggior successo saranno quelle che in futuro riusciranno ad adattarsi meglio al loro ambiente. Supponendo che questi sistemi debbano evolversi per sopravvivere, il progetto iniziale è importante ma, nel lungo termine, i meccanismi per il cambiamento lo saranno di più: essi sono noti come il livello verticale di amministrazione (\emph{governance}). Ci sono due componenti critici della governance:
\begin{itemize}
	\item
	      Incentivo: ogni gruppo nel sistema ha i propri incentivi. Gli incentivi non sono sempre allineati al 100\% con quelli di tutti gli altri gruppi nel sistema. I gruppi proporranno nel tempo cambiamenti che sono vantaggiosi per loro. Gli organismi sono "di parte" quando si tratta della propria sopravvivenza. Ciò normalmente si manifesta in cambiamenti nella struttura retributiva, nella politica monetaria o negli equilibri di potere.

	\item
	      Coordinamento: poiché è improbabile che tutti i gruppi risultino completamente allineati sugli incentivi in ogni momento, la capacità di ciascun gruppo di coordinarsi attorno agli incentivi comuni è fondamentale per loro per produrre un cambiamento. Se un gruppo riesce a coordinarsi meglio di un altro, crea uno squilibrio di potere a proprio favore. In pratica, un fattore decisivo per la sopravvivenza di una blockchain è quanto coordinamento si può realizzare utilizzando la blockchain (ad es. votando le regole del sistema come in Tezos \cite{c34}, o addirittura ripristinando uno stato precedente della blockchain se gli azionisti di maggioranza non gradiscono una modifica), rispetto al coordinamento che deve essere realizzato fuori dalla blockchain (come ad es. i Bitcoin Improvement Proposals (BIPs)
	      \cite{c3}).

\end{itemize}

\subsection{Modelli Operazionali}
Le blockchain possono essere categorizzate come "non autorizzate" ed "autorizzate", a seconda di come sono gestite. Ad esempio, Bitcoin è priva di autorizzazione, il che significa che chiunque può creare un indirizzo e iniziare a interagire con la rete: in questo caso si parla di "costruire fiducia in mancanza di affidabilità". Al contrario, la blockchain autorizzata è un ecosistema chiuso e monitorato, dove l'accesso di ciascun partecipante è defnito, e differenziato in base al suo ruolo: in questi casi si parla di "costruire fiducia in bassa affidabilità".
Vi sono vantaggi e svantaggi in ciascun approccio. Ad ogni modo, tutte queste considerazioni si riducono a compromessi di progetto fondamentali tra fiducia, scalabilità, elaborazione e complessità. Ad esempio, Bitcoin ed Ethereum sono blockchain costruite su nodi non affidabili, perché la scalabilità è fortemente desiderata. Quindi, o è richiesta molta elaborazione (nel caso del PoW), oppure è necessario un meccanismo di consenso più sofisticato. Al contrario, Fabric \cite{c14} è una blockchain autorizzata in cui tutti i nodi sono considerati affidabili e hanno identità crittografiche, ad esempio, rilasciate grazie ai servizi di membri come il Public Key Infrastructure (PKI), il che li rende altamente scalabili con poca elaborazione e un meccanismo di consenso relativamente semplice.

\begin{table}[htp]%
	\caption{Proprietà delle Blockchain: Benefici per l'IoT}
	\label{table:BlockchainBenefits}\centering %
	\begin{tabular}{l|l}
		\hline
		Proprietà della Blockchain  & Benefici per l'IoT       \\
		\hline
		Decentralizzazione          & Stabilità, Privacy       \\
		Bizantine Fault Tolerance   & Disponibilità, Sicurezza \\
		Trasaprenza \& Immutabilità & Assicurazione di Fiduca  \\
		Programmabilità             & Estendibilità            \\
		\hline
	\end{tabular}
\end{table}
