\section{L'Internet of Things}
L'Internet of Things (IoT) sta emergendo rapidamente come manifestazione della visione di una società collegata in rete: qualsiasi cosa che può beneficiare di una connessione, viene connesso. Eppure, questa trasformazione su vasta scala rappresenta solo l'inizio. Il numero di dispositivi IoT è destinato a crescere del 21\% ogni anno, raggiungendo i 18 miliardi nel 2022 \cite{c10}, mentre il mercato globale dell'IoT è destinato a passare dai 170 miliardi di dollari del 2017 a 560 miliardi di dollari entro il 2022 \cite{c15}, con un tasso di crescita annua del 26,9\%. Sebbene molti esperti dell'industria e consumatori entusiasti hanno definito l'IoT come la prossima rivoluzione industriale o il prossimo internet, ci sono tre problemi principali che frenano in maniera massiccia lo sviluppo e l'adozione dell'IoT.

\subsection{Il problema della scalabilità}
La maggior parte dei dispositivi IoT sono ad oggi connessi e controllati in maniera centralizzata. I dispositivi IoT sono connessi ad infrastrutture di back-end, su servizi cloud pubblici oppure localmente all'interno di server farm, per trasmettere dati oppure ricevere comandi di controllo.
Attualmente, la dimensione dell IoT è strozzata dal livello di scalabilità ed elasticità di queste infrastrutture di back-end, server e data center. E' improbabile che il costo operativo sostanzialmente elevato necessario per scalare l'IoT sia coperto dai profitti della vendita dei dispositivi. Di conseguenza, molti fornitori IoT non riescono a proporre dispositivi economicamente vantaggiosi ed applicazioni che siano abbastanza scalabili ed affidabili per scenari reali.

\subsection{Mancanza di Privacy}
Si prevede che l'IoT permetterà la partecipazione di massa degli utenti finali a servizi mission critical come l'energia, la  mobilità, la stabilità legale e democratica. Le sfide per la privacy hanno origine dal fatto che l'IoT interagisce con il mondo fisico in modi diretti e automatici, e la quantità di dati raccolti aumenterà notevolmente man mano che si diffonderà. Alcune  delle minacce alla privacy comuni, come elencate in \cite{c37}, sono:

\begin{enumerate}

	\item
	      Identificazione: associare un identificatore  persistente, ad es. un nome e un indirizzo o uno pseudonimo di qualsiasi tipo, con un individuo;
	\item
	      Localizzazione e tracciamento: ottenere la posizione di un individuo attraverso diversi mezzi;
	\item
	      Profilazione: Compilare fascicoli informativi sulle persone per dedurrne gli interessi per
	      associazione con altri profili e fonti di dati;
	\item
	      Interazione e presentazione che violano la privacy: trasmettere informazioni private attraverso un mezzo pubblico e nel processo rivelarle ad un pubblico indesiderato;
	\item
	      Transizioni del ciclo di vita: i dispositivi spesso memorizzano enormi quantità di dati sulla propria storia durante l'intero ciclo di vita, che potrebbero trapelare durante i cambiamenti della sfera di controllo nel ciclo di vita di un dispositivo;
	\item
	      Attacco all'inventario: raccolta non autorizzata di informazioni sull'esistenza e sulle caratteristiche degli oggetti personali, ad esempio, i ladri d'appartamento potrebbero utilizzare l'inventario dati per controllare la proprietà, e individuare un momento sicuro per entrare;
	\item
	      Collegamento: collegamento di sistemi diversi precedentemente separati in modo tale che la combinazione delle fonti di dati riveli informazioni (vere o errate) che il soggetto non aveva rivelato alle fonti isolate e, soprattutto, che non intendeva rivelare.
\end{enumerate}
Tutte queste tipiche minacce alla privacy sono dovute alla divulgazione indesiderata dei dati a livello del dispositivo, oppure  durante la comunicazione, o più spesso alla divulgazione dei dati nella parte centralizzata della rete.

\subsection{Mancanza di Valore Funzionale}
La maggior parte delle soluzioni IoT esistenti non crea valore significativo. Il semplice fatto di "essere connesso" rappresenta al momento la proposta di valore più utilizzata. Tuttavia, abilitarne semplicemente la connettività non rende un dispositivo intelligente, o utile. La maggior parte del valore che l'IoT produce è dovuto all'interazione, alla cooperazione, ed infine al coordinamento autonomo di entità eterogenee. Alcune buone analogie sono: le singole cellule che cooperano per costruire gli organismi multicellulari, gli insetti che insieme costruiscono società, oppure gli uomini che costituiscono città e stati. Grazie alla cooperazione, tutti questi individui si uniscono per costruire qualcosa che ha un valore maggiore rispetto a tutti loro presi isolatamente. Sfortunatamente, secondo \cite{c29}, l'85\% dei dispositivi obsoleti non ha la capacità di interagire o cooperare con altri dispositivi, a causa di problemi di compatibilità. La condivisione dei dati e le indicazioni operative per il business sono quasi irrealizzabili.
